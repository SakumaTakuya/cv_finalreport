\section{結論}\label{sec:conc}
本レポートはスポーツ中継のバーチャル広告などで使用される、
画像の差し替え技術を利用したアプリケーションを作成した。

SIFTを利用した特徴量によって並進・スケール・回転にある程度
頑健に置換できることを示した。
また、動画を対象としているという特徴を活かして、
精度を保つ、あるいは向上させながらも80\%程度の計算時間で実行できた。


今後の展望としては、次のような工夫が考えられる。
\begin{itemize}
    \item 特徴点検出アルゴリズムの検討
    \item 処理の高速化
\end{itemize}

現在の実装ではSIFT特徴量を利用しているが、
このアルゴリズムにはライセンスが必要であり、
商用のアプリケーションとしてリリースするには向いていない。
この問題を解決するアルゴリズムとして、AKAZE\cite{akaze}
などのアルゴリズムが提案されている。
AKAZEは商用利用できるだけでなく、SIFTで使用されている、
スケールスペースではガウシアンフィルタが
等方的に作用するために、エッジもぼやかしてしまうという欠点を
解決できるとされている\cite{akaze}。

また、本実装はPythonを利用しているために、
C++等の低級言語と比較して実行速度に難がある。
この問題を解決するために、
本実装ではできるだけfor文を使用せずに、
テンソル演算を活用した。
テンソル演算を活用できるということは
インデックス間で処理が独立していることを示しており、
GPGPUやOpenMPなどを活用できると言える。

